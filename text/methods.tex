\section{Methods}\label{methods}

In this section the underlying concepts and technologies are described. It starts with a brief description of XNAT and OpenStack. For further information about these systems and how they are integrated in the platform see~\cite{wu14}. The section continues with introducing the concept of workflows, the functionalities of WS-PGRADE/gUSE and the DCI-Bridge, which is an important part of gUSE. The last subsection describes a sample application, which has been used throughout the project.

\subsection{XNAT}\label{xnat}

TODO

\subsection{OpenStack}\label{openstack}

TODO

\subsection{Workflows}\label{workflows}

Applications consist of executable programs.
They usually take input, which can be in the form of command-line parameters or files and produce output files.
Some programs depend on the output of other programs, so that a mechanism is needed to ensure that all required inputs have been produced successfully before the program runs.
The workflow itself is a description or recipe how to link the executables together and are often stored in xml formats.
In terms of a workflow, every program is described as job, which can consist of one executable file or as in gUSE can be a zip file containing several files.
Every job has input and output ports, where each port has a number to identify it.
There can only be one file assigned to each port, which is then used to execute the program.
Every output port of one job can be linked to an input port of another program.
A system like gUSE that implements workflows ensures that the all necessary inputs are available before the job is started.
The produced data is then transported to the expected location where next job can see it.
When a job fails, it can resume this single job after the problem has been fixed.
There is no need to process all the successfully finished jobs again.
One of the most important feature of the ports is the support of naming conventions.
An input file assigned to a port is renamed, so that the program can find it in the storage.
The program also produces files that are named in a special way.
This name must be specified for the output port, so that the workflow system can keep track of the produced files.
Different jobs that have linked outputs and inputs can follow different naming conventions.

\subsection{WS-PGRADE/gUSE}\label{guse}

Grid and cloud User Support Environment (gUSE) is an opensource workflow management system developed at the Laboratory of Parallel and Distributed Systems in Hungaria.
WS-PGRADE is built on top of gUSE and is a collection of several Java Enterprise applications that are combined to form a full featured e-science portal.
It provides a user-interface based on the opensource portal software liferay.


\subsection{DCI-Bridge}\label{dci}

TODO

\subsection{Sample Applications}\label{applications}

TODO

\subsection{Workflow Repositories}\label{repositories}

TODO