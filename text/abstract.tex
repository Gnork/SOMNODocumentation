\begin{abstract}
In sleep research multicentric studies produce a large amount of data like electrocardiography scans.
Various processing algorithms can be applied to these scans.

The QMROCT project introduced a cloud based data processing infrastructure with XNAT as data archive.
For the SOMNO.Netz sleep research project the workflow management system WS-PGRADE/gUSE is introduced to the architecture.
It supports more complex algorithms with multiple dependent jobs and is able to control resources of the OpenStack cloud.
Since WS-PGRADE/gUSE was originally developed for grid infrastructures, this paper describes how to set it up in a cloud based system and evaluates the productive usage in several aspects.

The paper concludes that WS-PGRADE/gUSE is generally working in cloud system, but is not able to handle errors produced by the cloud which results in instabilities and lacks of cloud specific features.
\end{abstract}
