\section{Discussion}\label{discussion}

WS-PGRADE/gUSE is a workflow management system, that has been used to execute applications in grid infrastructures successfully.
By porting this functionality to a cloud infrastructure new challenges arise for data and error handling.

In the current state WS-PGRADE/gUSE has no cloud specific features.
The data model of sending files between gUSE server and the cloud infrastructure has not been adapted for cloud architectures.
The performance could be improved significantly by reducing the amount of data transfers.
For WS-PGRADE/gUSE the cloud is just another grid and the characteristics of cloud infrastructures are being ignored.

Since gUSE is able to perform a lot of tasks by providing a complex workflow system it would be great to use it in the QMROCT/SOMNO.Netz data processing infrastructure.
Unfortunately gUSE is only as stable as the underlying cloud and not able to handle any errors or delays caused by the cloud.
Therefore the system becomes unreliable.
Furthermore gUSE needs to be restarted as soon as unexpected errors occur.
It is the only way of restoring a stable system environment.

Due to this serious issues it is not possible to use WS-PGRADE/gUSE for a productive cloud-based architecture right now.
In order to improve the system, it is important to further investigate the given issues and to report back to the developers which improvements are necessary before a productive usage is possible.