\section{Introduction}\label{introduction}

In sleep research large amounts of patient data are collected during sleep studies, also called polysomnographies.
In the course of the SOMNO.Netz project~\cite{krefting13} a data archiving and processing infrastructure is provided, to manage the data of 300 sleep laboratories.
The infrastructure is based on XNAT as an archiving tool for medical data and an OpenStack cloud for scalable data processing.
This architecture has been introduced by Wu et al.~\cite{wu14} in the scope of the QM-ROCT project for medical image data, but is also suitable for biosignal processing like electrocardiography scans in sleep research.
The processing itself is done by Matlab algorithms executed on virtual machines in the cloud. These VMs can be launched and shut down on demand.
The current system architecture relies on various shell scripts, controlling the procedure and connect to the resources using REST \cite{richardson07} calls.
This is a lightweight solution, which provides a fast and straight forward process, but lacks in error handling and a more sophisticated resource management.
Especially algorithms depending on the output of a previous analysis are not yet supported.

In a collaboration with researchers from the ER-flow project at the AMC Amsterdam, we introduce the workflow management system WS-PGRADE/gUSE~\cite{balasko13} to the server architecture. This system has been used for grid-based science gateways, as developed by Shahand et al.~\cite{shahand13}, successfully.
This paper evaluates the usage of WS-PGRADE/gUSE in cloud-based architectures and the improvements that should be achieved for the platform as follows:

\begin{itemize}
\item Limiting the access to cloud resources
\item Optimizing the usage of limited cloud resources
\item Error handling and troubleshooting for unexpected cloud problems
\item Reliability of WS-PGRADE/gUSE in combination with the cloud
\item Responsiveness under high load
\item Overhead added by WS-PGRADE/gUSE
\item Starting workflows automatically via command-line interfaces
\item Can the system be used in productive environments?
\item Are workflows valuable for the platform?
\end{itemize}